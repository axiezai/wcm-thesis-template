\section{First Section}
% Replace these with your actual contents
\lipsum[1-2]

\section{Examples}
Here are some useful examples. Notice how we can do sections, subsections, and even subsubsections to divide up your text. All all sections can be labeled, and your table of contents will be generated automatically.

\subsection{Equation}
For more math examples, simply Google for \textit{overleaf}'s math documentations.
\begin{equation}
k_1=\frac{\omega }{c({1/\varepsilon_m + 1/\varepsilon_i})^{1/2}}=k_2=\frac{\omega
sin(\theta)\varepsilon_{air}^{1/2}}{c}
\end{equation}

\lipsum[3]

\subsection{Figure}
\begin{figure}
\centering
\includegraphics[width=0.75\textwidth]{../figures/octocat.png}
\caption{GitHub's Octocat}
\caption*{The previous title will show up in the list of figures automatically, this longer caption however, will not.}
\label{fig:octocat}
\end{figure}

\lipsum[4]

\subsection{Table}
% Use the same caption* and caption 
\begin{table}
 \caption{Sample table title}
 \caption*{See example from README for details on how to add larger tables that do not fit in margins if needed}
  \centering
  \begin{tabular}{lll}
    \toprule
    \multicolumn{2}{c}{Part}                   \\
    \cmidrule(r){1-2}
    Name     & Description     & Size ($\mu$m) \\
    \midrule
    Dendrite & Input terminal  & $\sim$100     \\
    Axon     & Output terminal & $\sim$10      \\
    Soma     & Cell body       & up to $10^6$  \\
    \bottomrule
  \end{tabular}
  \label{tab:table}
\end{table}

\lipsum[5]

\section{Conclusion}
\lipsum[6]
